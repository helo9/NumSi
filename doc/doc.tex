\documentclass[10pt,a4paper]{article}
\usepackage[utf8]{inputenc}
\usepackage[german]{babel}
\usepackage{amsmath}
\usepackage{amsfonts}
\usepackage{amssymb}

\title{doc}

\begin{document}
%\maketitle
%\tableofcontents
\section{Finite-Volumen-Methode für strukturiertes Vierecksgitter}
\subsection{Diskretisierung diffusiver Flüsse}
Nur Ost- und Nordseite werden diskretisiert. Diffusive Flüsse auf West- und Südseite werden außer am Rand mithilfe der Flüsse durch Ost- und Nordseite der benachbarten Kontrollvolumen bestimmt.
\subsubsection{Nord}
\begin{align}
F_n^D &\approx D_n ( \phi_N-\phi_P )+N_n(\phi_{ne} - \phi_{nw}) \\
D_n &= \alpha \frac{(x_{ne}-x_{nw})^2 +(y_{ne}-y_{nw})^2}{(x_{ne}-x_{nw})(y_N-y_P)-(y_{ne}-y_{nw})(x_N-x_P)} \\
N_n &= \alpha \frac{(y_N-y_P)(y_{nw}-y_{ne})+(x_N-x_P)(x_{nw}-x_{ne})}{(x_{ne}-x_{nw})(y_N-y_P)-(y_{ne}-y_{nw})(x_N-x_P)} \\
\end{align}
\subsubsection{Ost}
\begin{align}
F_e^D &\approx D_e ( \phi_E-\phi_P )+N_e(\phi_{ne} - \phi_{se}) \\
D_e &= \alpha \frac{(x_{ne}-x_{se})^2 +(y_{ne}-y_{se})^2}{(x_{ne}-x_{se})(y_E-y_P)-(y_{ne}-y_{se})(x_E-x_P)} \\
N_e &= \alpha \frac{(y_E-y_P)(y_{ne}-y_{se})+(x_E-x_P)(x_{ne}-x_{se})}{(y_{ne}-y_{se})(x_E-x_P)-(x_{ne}-x_{se})(y_E-y_P)}
\end{align}
\subsubsection{Eckwerte}
Die Eckwerte werden mithilfe der vier benachbarten Werte interpoliert. Als Gewichtungsfaktor $\gamma$ wird  der inverse Abstand verwendet.
\begin{equation}
\phi_{ne} = \frac{\sum_i \gamma_i \phi_i}{\sum_i \gamma_i}
\end{equation}
\subsection{Diskretisierung diffusiver Flüsse am Rand}
\subsection{Dirichlet}
\subsubsection{Nord}
\begin{equation}
F_n^D = -\alpha \frac{(x_{ne}-x_{nw})^2+(y_{ne}-y_{nw})^2}{(x_{ne}-x_{nw})(y_{nm}-y_P)-(y_{ne}-y_{nw})(x_{nm}-x_P)}(\phi_{nm}-\phi_P)
\end{equation}
\subsubsection{Ost}
\begin{equation}
F_e^D = -\alpha \frac{(x_{ne}-x_{se})^2+(y_{ne}-y_{se})^2}{(x_{em}-x_P)(y_{ne}-y_{se})-(y_{em}-y_P)(x_{ne}-x_{se})}(\phi_{em}-\phi_P)
\end{equation}
\subsubsection{Süd}
\begin{equation}
F_s^D = -\alpha \frac{(x_{se}-x_{sw})^2+(y_{se}-y_{sw})^2}{(x_{se}-x_{sw})(y_P-y_{sm})-(y_{se}-y_{sw})(x_P-x_{sm})}(\phi_{sm}-\phi_P)
\end{equation}
\subsubsection{West}
\begin{equation}
F_w^D = -\alpha \frac{(x_{nw}-x_{sw})^2+(y_{nw}-y_{sw})^2}{(x_P-x_{wm})(y_{nw}-y_{sw})-(y_P-y_{wm})(x_{nw}-x_{sw})}(\phi_P-\phi_{wm})
\end{equation}
\subsection{Neumann}
\subsubsection{Nord}
\subsubsection{Ost}
\subsubsection{Süd}
\subsubsection{West}



\section{Anwednung auf Wärmeleitung}

\begin{equation}
-\kappa \frac{\partial^2 T}{\partial x^2} -\kappa \frac{\partial^2 T}{\partial x^2} = \rho q 
\end{equation}

\end{document}